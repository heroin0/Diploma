\documentclass[a4paper,14pt,oneside]{book}
%\usepackage[french,latin,english,russian]{babel}

\usepackage[english,russian]{babel}   %% загружает пакет многоязыковой вёрстки
\usepackage{fontspec}      %% подготавливает загрузку шрифтов Open Type, True Type и др.
\defaultfontfeatures{Ligatures={TeX},Renderer=Basic}  %% свойства шрифтов по умолчанию
\setmainfont[Ligatures={TeX,Historic}]{Times New Roman} %% задаёт основной шрифт документа
\setsansfont{Comic Sans MS}                    %% задаёт шрифт без засечек
\setmonofont{Courier New}
\usepackage{indentfirst}
\frenchspacing   % интервалы между словами и предложениями - одинаковые

\clubpenalty=9999        %  без висячих строк
\widowpenalty=9999    %  без висячих строк

\usepackage[backend=biber,style=gost-numeric, % стиль цитирования и библиографии
sorting=none,     %  в порядке ссылок из текста,
language=auto, % получение языка из babel
%firstinits=false,   % ФИО полностью
babel=other % многоязычная библиография
]{biblatex}
\addbibresource{library.bib} % библиографическая база данных

\NewBibliographyString{langjapanese}     %  bug в gost
\NewBibliographyString{fromjapanese}    %  bug  в  gost

%\DefineBibliographyStrings{english}{pages={p.}}

\DefineBibliographyExtras{russian}{\protected\def\bibrangedash{\textendash}}    % EN DASH в страницах:  8-9  (черточка)

\DefineBibliographyExtras{russian}{
    \renewcommand*{\newblockpunct}{\addperiod\addnbspace\textendash\space\bibsentence}%  EN DASH между блоками (черточка)
}
%======================


%%% Работа с русским языком
\usepackage{cmap}                   % поиск в PDF
\usepackage{mathtext}               % русские буквы в фомулах

%%% Дополнительная работа с математикой
\usepackage{amsmath,amsfonts,amssymb,amsthm,mathtools} % AMS
\usepackage{icomma} % "Умная" запятая: $0,2$ --- число, $0, 2$ --- перечисление

%% Свои команды
\DeclareMathOperator{\sgn}{\mathop{sgn}}

%% Перенос знаков в формулах (по Львовскому)
\newcommand*{\hm}[1]{#1\nobreak\discretionary{}
{\hbox{$\mathsurround=0pt #1$}}{}}

%%% Работа с картинками
\usepackage{graphicx}  % Для вставки рисунков
\graphicspath{{/}}  % папки с картинками
\setlength\fboxsep{3pt} % Отступ рамки \fbox{} от рисунка
\setlength\fboxrule{1pt} % Толщина линий рамки \fbox{}
\usepackage{wrapfig} % Обтекание рисунков текстом

%%% Работа с таблицами
\usepackage{array,tabularx,tabulary,booktabs} % Дополнительная работа с таблицами
\usepackage{longtable}  % Длинные таблицы
\usepackage{multirow} % Слияние строк в таблице

%%% Теоремы
\theoremstyle{plain} % Это стиль по умолчанию, его можно не переопределять.
\newtheorem{theorem}{Теорема}[section]
\newtheorem{proposition}[theorem]{Утверждение}
 
\theoremstyle{definition} % "Определение"
\newtheorem{corollary}{Следствие}[theorem]
\newtheorem{problem}{Задача}[section]
 
\theoremstyle{remark} % "Примечание"
\newtheorem*{nonum}{Решение}

%%% Программирование
%\usepackage{etoolbox} % логические операторы

%%% Страница
\usepackage{extsizes} % Возможность сделать 14-й шрифт
\usepackage{geometry} % Простой способ задавать поля
    \geometry{top=20mm}
    \geometry{bottom=20mm}
    \geometry{left=20mm}
    \geometry{right=10mm}
 %
%\usepackage{fancyhdr} % Колонтитулы
  %  \pagestyle{fancy}
    %\renewcommand{\headrulewidth}{0mm}  % Толщина линейки, отчеркивающей верхний колонтитул
%    \lfoot{Нижний левый}
%    \rfoot{Нижний правый}
%    \rhead{Верхний правый}
%    \chead{Верхний в центре}
%    \lhead{Верхний левый}
    % \cfoot{Нижний в центре} % По умолчанию здесь номер страницы

\usepackage{setspace} % Интерлиньяж
%\onehalfspacing % Интерлиньяж 1.5
%\doublespacing % Интерлиньяж 2
%\singlespacing % Интерлиньяж 1

\usepackage{lastpage} % Узнать, сколько всего страниц в документе.

\usepackage{soul} % Модификаторы начертания, highlight text

\usepackage{hyperref}
\usepackage[usenames,dvipsnames,svgnames,table,rgb]{xcolor}
\hypersetup{                % Гиперссылки
    unicode=true,           % русские буквы в раздела PDF
    pdftitle={Заголовок},   % Заголовок
    pdfauthor={Автор},      % Автор
    pdfsubject={Тема},      % Тема
    pdfcreator={Создатель}, % Создатель
    pdfproducer={Производитель}, % Производитель
    pdfkeywords={keyword1} {key2} {key3}, % Ключевые слова
    colorlinks=true,        % false: ссылки в рамках; true: цветные ссылки
%    linkcolor=[rgb]{0.047,0.023,0.121},          % внутренние ссылки  , оглавление здесь 
%    linkcolor=[rgb]{0.2,0.1,0.5},          % внутренние ссылки  , оглавление здесь
     linkcolor=[rgb]{0,0,1},          % внутренние ссылки  , оглавление здесь
%    linkcolor=[rgb]{0.1016,0.074,0.1836},          % внутренние ссылки  , оглавление здесь
%    citecolor=green,        % на библиографию
    citecolor=[rgb]{0,0,1},        % на библиографию
%    filecolor=magenta,      % на файлы
    filecolor=[rgb]{0,0,1},      % на файлы
    urlcolor=[rgb]{0,0,1}           % на URL
%    urlcolor=cyan           % на URL
}
%%%%%%%%%%%%%%%%%% меняю список литературы
\makeatletter
\renewcommand{\@biblabel}[1]{#1.\hfil}
\makeatother
\addto\captionsrussian{\def\refname{Список использованных источников}}
\renewcommand{\bibname}{Список использованных/par/centerline{источников}}
%%%%%%%%%%%%%%%%%%%%%%%%%%%%%%%%%%%
%\renewcommand{\familydefault}{\sfdefault} % Начертание шрифта

\usepackage{multicol} % Несколько колонок

%\author{\LaTeX{} в Вышке}
%\title{3.2 Оформление документа в целом}
%\date{\today}
\usepackage[justification=justified,singlelinecheck=false]{caption}  % параметры - выравнивание влево
\usepackage{titlesec}    % formatting chapter title
\titleformat{\chapter}[hang] 
{\normalfont\Large\bfseries}{\thechapter}{1em}{}   %  word chapter and chapter name in one line
% remove word chapter and decreas font size
% сплошная нумерация таблиц и рисунков
\usepackage{chngcntr}   
\counterwithout{figure}{chapter}
\counterwithout{table}{chapter}

%\usepackage{geometry}
%\geometry{verbose,tmargin=20mm,bmargin=20mm,lmargin=30mm,rmargin=15mm}  %  margins

% списки без вертикальных интервалов по Столярову
\newenvironment{compactlist}{
\begin{list}{{$\bullet$}}{
\setlength\partopsep{0pt}
\setlength\parskip{0pt}
\setlength\parsep{0pt}
\setlength\topsep{0pt}
\setlength\itemsep{0pt}
}
}{
\end{list}
}
\renewcommand{\bibname}{Список использованных/par/centerline{источников}}

