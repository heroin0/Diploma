\chapter*{ВВЕДЕНИЕ}
\addcontentsline{toc}{chapter}{ВВЕДЕНИЕ}
В ходе развития компьютерных наук человечеств встретилось с классом NP-полных задач. Такие задачи решаются алгоритмически за недетерминированное полиномиальное время, что существенно затрудняет поиск решения таких задач в приемлемые сроки.
 Одной из таких задач является задача о рюкзаке(Knapsack problem) и её модификации.
 %объект и предмет исследования; 
 Объектом исследования даной работы является задача о многомерном рюкзаке (Multidimensional knapsack problem).
 Предмет исследования - решение задачи о многомерном рюкзаке с использованием генетического алгоритма.
%актуальность выбранной темы;  
\\Многие прикладные проблемы могут быть формализованы в виде рассматриваемой задачи. Примерами таких проблем являются размещение процессоров
и баз данных в системе распределенных вычислений (см. \cite{Гэвиш1982}), погрузка груза и контроль бюджета (см. \cite{Ших1979}), задачи раскройки (см. \cite{Гилмор1966}) и др. 
Решение этих проблем обуславливает актуальность решения задачи о рюкзаке.

%цель и задачи исследования; 
Цель исследования - решить задачу о многомерном рюкзаке с использованием генетического алгоритма
Для достижения цели были поставлены следующие задачи
\begin{itemize}
\item Исследовать генетические алгоритмы в применении к NP-полным задачам
\item Спроектировать и реализовать генетический алгоритм, решающий задачу о многомерном рюкзаке
\item Провести оценку эффективности генетического алгоритма в решении поставленной задачи, используя набор готовых тестов
\end{itemize} 
Для реализации алгоритма был выбран язык C\# ввиду удобства и простоты работы.
%методы и подходы, применяемые при исследовании;(генетика)  
%выводы  о  конкретных  результатах  работы;  (получили такие-то результаты) 