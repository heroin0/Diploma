\chapter{Вторая глава}

\section{Результаты реализации алгоритма}

Для оценки эффективности работы алгоритма было проведено его тестирование на различных наборах тестов.
Первый набор тестов взят из книги Петерсена(см. \cite{Петерсен1967}) и содержит 7 задач. Условия этих задач(см таблицу \ref{table1}) позволяют проверить эффективность простейшей версии генетического алгоритма без модификаций и оценить её эффективность. %оформление цитаты

\begin{table}[ht]%таблица выглядит очень странно ---
\centering
\caption{Параметры первого набора тестов}
\label{table1}
\begin{tabular}{|c|c|c|}
\hline
№ задачи & Размерность & Количество предметов \\ \hline
1        & 6           & 10                   \\ \hline
2        & 10          & 10                   \\ \hline
3        & 15          & 10                   \\ \hline
4        & 20          & 10                   \\ \hline
5        & 28          & 10                   \\ \hline
6        & 39          & 5                    \\ \hline
7        & 50          & 5                    \\ \hline
\end{tabular}
\end{table}  
Второй набор тестов взят из статьи Чу (см \cite{Чу1998}) и содержит в себе 30 задач с одинаковыми параметрами: размерность рюкзака равна 5, рассматривается 100 различных предметов.

%что должно быть здесь
%все с клетчатого листочка
%старая программа, таблицы из презентации
%новая програииа, новая таблица с тестовыми данными
%график новых данных
%откуда брали тестовые данные.