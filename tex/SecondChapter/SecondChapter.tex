\chapter{Вторая глава}

\section{Наборы тестов}

Для оценки эффективности работы алгоритма было проведено его тестирование на различных наборах тестов.
Первый набор тестов взят из книги Петерсена(см. \cite{Петерсен1967}) и содержит 7 задач. Условия этих задач(см таблицу \ref{table1}) позволяют проверить эффективность простейшей версии генетического алгоритма без модификаций и оценить её эффективность. %оформление цитаты

\begin{table}[ht]%таблица выглядит очень странно ---
\centering
\caption{Параметры первого набора тестов}
\label{table1}
\begin{tabular}{|c|c|c|}
\hline
№ задачи & Размерность & Количество предметов \\ \hline
1        & 6           & 10                   \\ \hline
2        & 10          & 10                   \\ \hline
3        & 15          & 10                   \\ \hline
4        & 20          & 10                   \\ \hline
5        & 28          & 10                   \\ \hline
6        & 39          & 5                    \\ \hline
7        & 50          & 5                    \\ \hline
\end{tabular}
\end{table}  

Второй набор тестов взят из статьи Чу (см \cite{Чу1998}) и содержит в себе 30 задач с одинаковыми параметрами: размерность рюкзака равна 5, рассматривается 100 различных предметов.
Для каждого набора тестов известно лучшее решение, эффективность алгоритма оценивалась по скорости поиска решения и числу итераций. 
\section{Первый набор тестов}
Рассмотрим результаты работы алгоритма без модификаций на первом наборе тестов. Экпериментальным образом были подобраны количество конфигураций в пуле(10) и процент мутирующих особей(20%). Рассматриваются медианные значения за 10 запусков. 
\subsection{Мутация в одной позиции}
Данная мутация показала себя эффективной с точки зрения решения задачи - для всех возможных вариантов скрещивания во всех тестах найден оптимальный результат(см таблицы \ref{SingleMutSingleCross}, \ref{SingleMutDoubleCross}, \ref{SingleMutBitCross})
При сравнении результатов скрещивания можно заметить, что алгоритм побитового скрещивания в большинстве случаев несколько уступает в скрости другим алгоритмам.
\begin{table}[ht]
\centering
\caption{Одноточечная мутация, одноточечное скрещивание}
\label{SingleMutSingleCross}
\begin{tabular}{|c|c|c|}
\hline
№ теста             & Итерации            & Время, мс            \\ \hline
1                   & 6                   & 8                    \\ \hline
2                   & 114340              & 2694                 \\ \hline
3                   & 712                 & 18                   \\ \hline
4                   & 4022                & 92                   \\ \hline
5                   & 3049                & 97                   \\ \hline
6                   & 400974              & 13481                \\ \hline
7                   & 9343432             & 396644               \\ \hline
\end{tabular}
\end{table}

\begin{table}[ht]
\centering
\caption{Одноточечная мутация, двуточечное скрещивание}
\label{SingleMutDoubleCross}
\begin{tabular}{|c|c|c|}
\hline
№ теста & Итерации & Время, мс \\ \hline
1       & 21       & 9         \\ \hline
2       & 70930    & 1697      \\ \hline
3       & 901      & 22        \\ \hline
4       & 2092     & 67        \\ \hline
5       & 2672     & 95        \\ \hline
6       & 23261    & 818       \\ \hline
7       & 4770456  & 199461    \\ \hline
\end{tabular}
\end{table}

\begin{table}[ht]
\centering
\caption{Одноточечная мутация, побитовое скрещивание}
\label{SingleMutBitCross}
\begin{tabular}{|c|c|c|}
\hline
№ теста & Итерации & Время. мс \\ \hline
1       & 3        & 9         \\ \hline
2       & 182633   & 3301      \\ \hline
3       & 1455     & 49        \\ \hline
4       & 7869     & 219       \\ \hline
5       & 2198     & 74        \\ \hline
6       & 289379   & 100133    \\ \hline
7       & 348826   & 139403    \\ \hline
\end{tabular}
\end{table}

\subsection{Инверсионная мутация}
Инверсионная мутация в контексте данной задачи оказалась менее эффективной, для тестов 5-7 решения не были найдены за 20 минут для всех вариантов скрещивания(см табл \ref{InvMutSingleCross}, \ref{InvMutDoubleCross}, \ref{InvMutBitCross}). Неэффективность такого варианта мутации может быть обусловлена применением алгоритма коррекции после мутации, что сводит на нет какой-либо положительный эффект от её применения.
Также можно отметить существенно более высокую скорость решения в сравнении с алгоритмом одноточечной мутации для теста 2.
\begin{table}[ht]
\centering
\caption{Инверсионная мутация, одноточечное скрещивание}
\label{InvMutSingleCross}
\begin{tabular}{|c|c|c|}
\hline
№ теста & Итерации & Время. мс \\ \hline
1       & 3        & 6         \\ \hline
2       & 10       & 2         \\ \hline
3       & 58       & 2         \\ \hline
4       & 9538     & 560       \\ \hline
5       & -        & -         \\ \hline
6       & -        & -         \\ \hline
7       & -        & -         \\ \hline
\end{tabular}
\end{table}

\begin{table}[ht]
\centering
\caption{Инверсионная мутация, двуточечное скрещивание}
\label{InvMutDoubleCross}
\begin{tabular}{|c|c|c|}
\hline
№ теста & Итерации & Время. мс \\ \hline
1       & 2        & 10        \\ \hline
2       & 16       & 1         \\ \hline
3       & 356      & 19        \\ \hline
4       & 7161     & 422       \\ \hline
5       & -        & -         \\ \hline
6       & -        & -         \\ \hline
7       & -        & -         \\ \hline
\end{tabular}
\end{table}

\begin{table}[ht]
\centering
\caption{Инверсионная мутация, побитовое скрещивание}
\label{InvMutBitCross}
\begin{tabular}{|c|c|c|}
\hline
№ теста & Итерации & Время. мс \\ \hline
1       & 10       & 10        \\ \hline
2       & 23       & 1         \\ \hline
3       & 1409     & 52        \\ \hline
4       & 246001   & 12189     \\ \hline
5       & -        & -         \\ \hline
6       & -        & -         \\ \hline
7       & -        & -         \\ \hline
\end{tabular}
\end{table}

\subsection{Промежуточные выводы}
Исходя из приведенных тестов, можно считать потвержденной принципиальную возможность реализованного алгоритма быть использованным для решения задачи о многомерном рюкзаке. Набилоее эффективной себя показало двуточечное скрещивание.
\section{Второй набор тестов}
Для проверки эффективности второго набора тестов были выбраны  параметры запуска алгоритма, указанные в таблице \ref{ParamTable}.

\begin{table}[ht]
\centering
\caption{Параметры запуска второго набора тестов}
\label{ParamTable}
\begin{tabular}{|c|c|}
\hline
Вид скрещивания                            & Двуточечное  \\ \hline
Вид мутации                                & Одноточечная \\ \hline
Размер пула генотипов                      & 10           \\ \hline
Процент мутирующих особей                  & 20\%         \\ \hline
Максимальное число итераций                & 2000000      \\ \hline
Величина отклонения для сброса алгоритма   & 0.01         \\ \hline
Количество итераций до полного перезапуска & 200000       \\ \hline
\end{tabular}
\end{table}
%на чем гоняли
Максимальное число итераций было введено с целью ограничить время работы алгоритма. В случае, если лучшее решение не найдено, программа выдает на печать расхождение лучшего из найденных рещений с текущим.\\
Рассмотрим результаты работы программы на втором наборе тестов \ref{BigResultsTable}. В 43.3% случаев генетический алгоритм решил задачу и во всех случаях расхождение найденного и действительного максимумов не превышает 100, разница 0,5 с опт результатом.

\begin{table}[]
\centering
\caption{My caption}
\label{my-label}
\begin{tabular}{|c|c|c|c|c|}
\hline
№ теста & Количество итераций & Время, с & Абсолютное расхождение с максимумом & Максимум \\ \hline
1       & -                   & 888      & 52                                  & 24381    \\ \hline
2       & 197915              & 44       & 0                                   & 24274    \\ \hline
3       & -                   & 915      & 13                                  & 23551    \\ \hline
4       & -                   & 812      & 54                                  & 23534    \\ \hline
5       & -                   & 860      & 44                                  & 23991    \\ \hline
6       & -                   & 970      & 36                                  & 24613    \\ \hline
7       & 1253381             & 619      & 0                                   & 25591    \\ \hline
8       & 83207               & 39       & 0                                   & 23410    \\ \hline
9       & 262645              & 255      & 0                                   & 24216    \\ \hline
10      & -                   & 963      & 76                                  & 24411    \\ \hline
11      & 189635              & 83       & 0                                   & 42757    \\ \hline
12      & -                   & 912      & 89                                  & 42545    \\ \hline
13      & -                   & 912      & 34                                  & 41968    \\ \hline
14      & -                   & 825      & 19                                  & 45090    \\ \hline
15      & -                   & 881      & 73                                  & 42218    \\ \hline
16      & 942021              & 484      & 0                                   & 42927    \\ \hline
17      & -                   & 864      & 29                                  & 42009    \\ \hline
18      & 4386                & 5        & 0                                   & 45020    \\ \hline
19      & -                   & 880      & 78                                  & 43441    \\ \hline
20      & -                   & 865      & 23                                  & 44554    \\ \hline
21      & -                   & 734      & 98                                  & 59822    \\ \hline
22      & 1735194             & 752      & 0                                   & 62081    \\ \hline
23      & -                   & 779      & 48                                  & 59802    \\ \hline
24      & 845324              & 428      & 0                                   & 60479    \\ \hline
25      & -                   & 812      & 46                                  & 61091    \\ \hline
26      & 908217              & 454      & 0                                   & 58959    \\ \hline
27      & 594220              & 281      & 0                                   & 61538    \\ \hline
28      & 36434               & 35       & 0                                   & 61520    \\ \hline
29      & -                   & 895      & 36                                  & 59453    \\ \hline
30      & 1423522             & 625      & 0                                   & 59965    \\ \hline
\end{tabular}
\end{table}
+ улучшения
%что должно быть здесь
%все с клетчатого листочка
%новая програииа, новая таблица с тестовыми данными
%график новых данных
%откуда брали тестовые данные.