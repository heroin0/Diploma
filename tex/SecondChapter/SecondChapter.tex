\chapter{Вторая глава}

\section{Наборы тестов}

Для оценки эффективности работы алгоритма было проведено его тестирование на различных наборах тестов.
Первый набор тестов взят из книги Петерсена(см. \cite{Петерсен1967}) и содержит 7 задач. Условия этих задач(см таблицу \ref{table1}) позволяют проверить эффективность простейшей версии генетического алгоритма без модификаций и оценить её эффективность. %оформление цитаты

\begin{table}[ht]%таблица выглядит очень странно ---
\centering
\caption{Параметры первого набора тестов}
\label{table1}
\begin{tabular}{|c|c|c|}
\hline
№ задачи & Размерность & Количество предметов \\ \hline
1        & 6           & 10                   \\ \hline
2        & 10          & 10                   \\ \hline
3        & 15          & 10                   \\ \hline
4        & 20          & 10                   \\ \hline
5        & 28          & 10                   \\ \hline
6        & 39          & 5                    \\ \hline
7        & 50          & 5                    \\ \hline
\end{tabular}
\end{table}  
Второй набор тестов взят из статьи Чу (см \cite{Чу1998}) и содержит в себе 30 задач с одинаковыми параметрами: размерность рюкзака равна 5, рассматривается 100 различных предметов.
Для каждого набора тестов известно лучшее решение, эффективность алгоритма оценивалась по скорости поиска решения в миллисекнудах и числу итераций. Для первого набора тестов рассматривались медианные значниия по 10 запускам 
\section{Первый набор тестов}
\subsection{Мутация в одной позиции}
\\Рассмотрим результаты работы алгоритма без модификаций на первом наборе тестов. Для одноточечной мутации(см таблицу \ref{tablePoint}) алгоритм находит целевое решение 

\begin{table}[ht]
\centering
\caption{Одноточечная мутация, одноточечное скрещивание}
\label{SingleMutSingleCross}
\begin{tabular}{|c|c|c|}
\hline
№ теста             & Итерации            & Время, мс            \\ \hline
1                   & 6                   & 8                    \\ \hline
2                   & 114340              & 2694                 \\ \hline
3                   & 712                 & 18                   \\ \hline
4                   & 4022                & 92                   \\ \hline
5                   & 3049                & 97                   \\ \hline
6                   & 400974              & 13481                \\ \hline
7                   & 9343432             & 396644               \\ \hline
\end{tabular}
\end{table}

\begin{table}[ht]
\centering
\caption{Одноточечная мутация, двуточечное скрещивание}
\label{SingleMutDoubleCross}
\begin{tabular}{|с|с|с|}
\hline
№ теста & Итерации & Время, мс \\ \hline
1       & 21       & 9         \\ \hline
2       & 70930    & 1697      \\ \hline
3       & 901      & 22        \\ \hline
4       & 2092     & 67        \\ \hline
5       & 2672     & 95        \\ \hline
6       & 23261    & 818       \\ \hline
7       & 4770456  & 199461    \\ \hline
\end{tabular}
\end{table}

\begin{table}[ht]
\centering
\caption{Одноточечная мутация, побитовое скрещивание}
\label{SingleMutBitCross}
\begin{tabular}{|c|c|c|}
\hline
№ теста & Итерации & Время. мс \\ \hline
1       & 3        & 9         \\ \hline
2       & 1826334  & 33017     \\ \hline
3       & 1455     & 49        \\ \hline
4       & 7869     & 219       \\ \hline
5       & 2198     & 74        \\ \hline
6       & 289379   & 10013     \\ \hline
7       & 348826   & 13940     \\ \hline
\end{tabular}
\end{table}
\subsection{Инверсионная мутация}
\\Половинная мутация в контексте данной задачи оказалась менее эффективной(см таблицу \ref{tableHalf})

\begin{table}[]
\centering
\caption{Инверсионная мутация, одноточечное скрещивание}
\label{InvMutSingleCross}
\begin{tabular}{|c|c|c|}
\hline
№ теста & Итерации & Время. мс \\ \hline
1       & 3        & 6         \\ \hline
2       & 10       & 2         \\ \hline
3       & 58       & 2         \\ \hline
4       & 9538     & 560       \\ \hline
5       & -        & -         \\ \hline
6       & -        & -         \\ \hline
7       & -        & -         \\ \hline
\end{tabular}
\end{table}

\begin{table}[]
\centering
\caption{Инверсионная мутация, двуточечное скрещивание}
\label{InvMutDoubleCross}
\begin{tabular}{|c|c|c|}
\hline
№ теста & Итерации & Время. мс \\ \hline
1       & 2        & 10        \\ \hline
2       & 16       & 1         \\ \hline
3       & 356      & 19        \\ \hline
4       & 7161     & 422       \\ \hline
5       & -        & -         \\ \hline
6       & -        & -         \\ \hline
7       & -        & -         \\ \hline
\end{tabular}
\end{table}

\begin{table}[]
\centering
\caption{Инверсионная мутация, побитовое скрещивание}
\label{InvMutBitCross}
\begin{tabular}{|c|c|c|}
\hline
№ теста & Итерации & Время. мс \\ \hline
1       & 10       & 10        \\ \hline
2       & 23       & 1         \\ \hline
3       & 1409     & 52        \\ \hline
4       & 246001   & 12189     \\ \hline
5       & -        & -         \\ \hline
6       & -        & -         \\ \hline
7       & -        & -         \\ \hline
\end{tabular}
\end{table}

\section{Второй набор тестов}

%что должно быть здесь
%все с клетчатого листочка
%старая программа, таблицы из презентации
%новая програииа, новая таблица с тестовыми данными
%график новых данных
%откуда брали тестовые данные.