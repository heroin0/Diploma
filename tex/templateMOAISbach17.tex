
\documentclass[a4paper,14pt,oneside]{book}
%\usepackage[french,latin,english,russian]{babel}

\usepackage[english,russian]{babel}   %% загружает пакет многоязыковой вёрстки
\usepackage{fontspec}      %% подготавливает загрузку шрифтов Open Type, True Type и др.
\defaultfontfeatures{Ligatures={TeX},Renderer=Basic}  %% свойства шрифтов по умолчанию
\setmainfont[Ligatures={TeX,Historic}]{Times New Roman} %% задаёт основной шрифт документа
\setsansfont{Comic Sans MS}                    %% задаёт шрифт без засечек
\setmonofont{Courier New}
\usepackage{indentfirst}
\frenchspacing   % интервалы между словами и предложениями - одинаковые

\clubpenalty=9999        %  без висячих строк
\widowpenalty=9999    %  без висячих строк

\usepackage[backend=biber,style=gost-numeric, % стиль цитирования и библиографии
sorting=none,     %  в порядке ссылок из текста,
language=auto, % получение языка из babel
%firstinits=false,   % ФИО полностью
babel=other % многоязычная библиография
]{biblatex}
\addbibresource{bibliogr.bib} % библиографическая база данных

\NewBibliographyString{langjapanese}     %  bug в gost
\NewBibliographyString{fromjapanese}    %  bug  в  gost

%\DefineBibliographyStrings{english}{pages={p.}}

\DefineBibliographyExtras{russian}{\protected\def\bibrangedash{\textendash}}    % EN DASH в страницах:  8-9  (черточка)

\DefineBibliographyExtras{russian}{
    \renewcommand*{\newblockpunct}{\addperiod\addnbspace\textendash\space\bibsentence}%  EN DASH между блоками (черточка)
}
%======================


%%% Работа с русским языком
\usepackage{cmap}                   % поиск в PDF
\usepackage{mathtext}               % русские буквы в фомулах

%%% Дополнительная работа с математикой
\usepackage{amsmath,amsfonts,amssymb,amsthm,mathtools} % AMS
\usepackage{icomma} % "Умная" запятая: $0,2$ --- число, $0, 2$ --- перечисление

%% Свои команды
\DeclareMathOperator{\sgn}{\mathop{sgn}}

%% Перенос знаков в формулах (по Львовскому)
\newcommand*{\hm}[1]{#1\nobreak\discretionary{}
{\hbox{$\mathsurround=0pt #1$}}{}}

%%% Работа с картинками
\usepackage{graphicx}  % Для вставки рисунков
%\graphicspath{{images/}{images2/}}  % папки с картинками
\setlength\fboxsep{3pt} % Отступ рамки \fbox{} от рисунка
\setlength\fboxrule{1pt} % Толщина линий рамки \fbox{}
\usepackage{wrapfig} % Обтекание рисунков текстом

%%% Работа с таблицами
\usepackage{array,tabularx,tabulary,booktabs} % Дополнительная работа с таблицами
\usepackage{longtable}  % Длинные таблицы
\usepackage{multirow} % Слияние строк в таблице

%%% Теоремы
\theoremstyle{plain} % Это стиль по умолчанию, его можно не переопределять.
\newtheorem{theorem}{Теорема}[section]
\newtheorem{proposition}[theorem]{Утверждение}
 
\theoremstyle{definition} % "Определение"
\newtheorem{corollary}{Следствие}[theorem]
\newtheorem{problem}{Задача}[section]
 
\theoremstyle{remark} % "Примечание"
\newtheorem*{nonum}{Решение}

%%% Программирование
%\usepackage{etoolbox} % логические операторы

%%% Страница
\usepackage{extsizes} % Возможность сделать 14-й шрифт
\usepackage{geometry} % Простой способ задавать поля
    \geometry{top=20mm}
    \geometry{bottom=20mm}
    \geometry{left=20mm}
    \geometry{right=10mm}
 %
%\usepackage{fancyhdr} % Колонтитулы
  %  \pagestyle{fancy}
    %\renewcommand{\headrulewidth}{0mm}  % Толщина линейки, отчеркивающей верхний колонтитул
%    \lfoot{Нижний левый}
%    \rfoot{Нижний правый}
%    \rhead{Верхний правый}
%    \chead{Верхний в центре}
%    \lhead{Верхний левый}
    % \cfoot{Нижний в центре} % По умолчанию здесь номер страницы

\usepackage{setspace} % Интерлиньяж
%\onehalfspacing % Интерлиньяж 1.5
%\doublespacing % Интерлиньяж 2
%\singlespacing % Интерлиньяж 1

\usepackage{lastpage} % Узнать, сколько всего страниц в документе.

\usepackage{soul} % Модификаторы начертания, highlight text

\usepackage{hyperref}
\usepackage[usenames,dvipsnames,svgnames,table,rgb]{xcolor}
\hypersetup{                % Гиперссылки
    unicode=true,           % русские буквы в раздела PDF
    pdftitle={Заголовок},   % Заголовок
    pdfauthor={Автор},      % Автор
    pdfsubject={Тема},      % Тема
    pdfcreator={Создатель}, % Создатель
    pdfproducer={Производитель}, % Производитель
    pdfkeywords={keyword1} {key2} {key3}, % Ключевые слова
    colorlinks=true,        % false: ссылки в рамках; true: цветные ссылки
%    linkcolor=[rgb]{0.047,0.023,0.121},          % внутренние ссылки  , оглавление здесь 
%    linkcolor=[rgb]{0.2,0.1,0.5},          % внутренние ссылки  , оглавление здесь
     linkcolor=[rgb]{0,0,1},          % внутренние ссылки  , оглавление здесь
%    linkcolor=[rgb]{0.1016,0.074,0.1836},          % внутренние ссылки  , оглавление здесь
%    citecolor=green,        % на библиографию
    citecolor=[rgb]{0,0,1},        % на библиографию
%    filecolor=magenta,      % на файлы
    filecolor=[rgb]{0,0,1},      % на файлы
    urlcolor=[rgb]{0,0,1}           % на URL
%    urlcolor=cyan           % на URL
}
%%%%%%%%%%%%%%%%%% меняю список литературы
\makeatletter
\renewcommand{\@biblabel}[1]{#1.\hfil}
\makeatother
\addto\captionsrussian{\def\refname{Список использованных источников}}
\renewcommand{\bibname}{Список использованных/par/centerline{источников}}
%%%%%%%%%%%%%%%%%%%%%%%%%%%%%%%%%%%
%\renewcommand{\familydefault}{\sfdefault} % Начертание шрифта

\usepackage{multicol} % Несколько колонок

%\author{\LaTeX{} в Вышке}
%\title{3.2 Оформление документа в целом}
%\date{\today}
\usepackage[justification=justified,singlelinecheck=false]{caption}  % параметры - выравнивание влево
\usepackage{titlesec}    % formatting chapter title
\titleformat{\chapter}[hang] 
{\normalfont\Large\bfseries}{\thechapter}{1em}{}   %  word chapter and chapter name in one line
% remove word chapter and decreas font size
% сплошная нумерация таблиц и рисунков
\usepackage{chngcntr}   
\counterwithout{figure}{chapter}
\counterwithout{table}{chapter}

%\usepackage{geometry}
%\geometry{verbose,tmargin=20mm,bmargin=20mm,lmargin=30mm,rmargin=15mm}  %  margins

% списки без вертикальных интервалов по Столярову
\newenvironment{compactlist}{
\begin{list}{{$\bullet$}}{
\setlength\partopsep{0pt}
\setlength\parskip{0pt}
\setlength\parsep{0pt}
\setlength\topsep{0pt}
\setlength\itemsep{0pt}
}
}{
\end{list}
}


\renewcommand{\bibname}{Список использованных/par/centerline{источников}}

\begin{document} % конец преамбулы, начало документа

%\maketitle

\begin{titlepage}
{\small
\centerline{Министерство образования и науки Российской Федерации}
\centerline{Федеральное государственное автономное образовательное учреждение}
\centerline{высшего образования}
\centerline{\normalsize<<Уральский федеральный университет}
\centerline{\normalsize имени первого Президента России Б.Н.Ельцина>>}
%\end{center}
\vskip0.2cm
\centerline{\normalsize Институт радиоэлектроники и информационных технологий --- РтФ}
\vskip0.2cm\centerline{\normalsize Департамент информационных технологий и автоматики}
}
\vskip1cm

\null\hfill
\begin{minipage}{0.6\textwidth}
\hfill ДОПУСТИТЬ К ЗАЩИТЕ В ГЭК \\ \\
РОП 02.03.03\rule[-1pt]{3.5cm}{0.4pt}
\\\phantom{XXXXXXXXXXXXXXXXXXX} В.И.~Белоусова
\\%{\underline{\hspace{3cm}}}$
%\hfill
%\hspace{4pt}
%$\underset{\text{(Ф.И.О.}}{\underline{\hspace{4.5cm}}}$\\
%\hfill$\ll$\rule[-1pt]{0.5cm}{0.4pt}$\gg$\rule[-1pt]{4cm}{0.4pt} 2016г.    % кавычки елочкой командой
\hfill <<10>> июня  2017г.
\end{minipage}\\
\vskip2cm
\centerline{\bf ТЕМА ВАШЕЙ ЗЕМЕЧАТЕЛЬНОЙ РАБОТЫ}
\vskip0.2cm
\centerline{{ВЫПУСКНАЯ КВАЛИФИКАЦИОННАЯ РАБОТА}}

%\centerline{Пояснительная записка}
%\centerline{ХХХХХХХХХХХХХХХХХХ}
\vskip5.5cm
\noindent
Научный руководитель: д.т.н., профессор А.П. Исаев\hfill $\underset{\text{(подпись)}}{\underline{\hspace{3cm}}}$\\
%Консультант: У.Н. Бабушкина \hfill $\underset{\text{(подпись)}}{\underline{\hspace{3cm}}}$\\
%Консультант: А.М. Сотников \hfill $\underset{\text{(подпись)}}{\underline{\hspace{3cm}}}$\\
\vskip0.2cm
Нормоконтролер: доц., к.ф.-м.н. А.Л. Крохин \hfill $\underset{\text{(подпись)}}{\underline{\hspace{3cm}}}$\\
\vskip0.2cm
Студент группы  РИ-430016\rule[-1pt]{1.5cm}{0.4pt} \hfill $\underset{\text{(подпись)}}{\underline{\hspace{3cm}}}$\\
\vfill
\centerline{Екатеринбург}
\centerline{2017}
\end{titlepage}

\pagestyle{plain}
\chapter*{РЕФЕРАТ}
%\addcontentsline{toc}{chapter}{РЕФЕРАТ}

\thispagestyle{empty}   % отменить вывод номера страницы
Данный шаблон, файл с расширением .tex, предназначен для того, чтобы внести в него изменения и получить персональную ВКР. В то же время он является набором примеров, как верстать отдельные элементы в Latex. Прилагаемый .PDF файл содержит краткую инструкцию по работе в Latex и ссылки на литературу.

Номер страницы реферата не печатается и эта страница не учитывается в нумерации страниц ВКР.

На страницах оглавления не выводятся номера, но они учитываются при нумерации страниц ВКР.

\cleardoublepage                       %  номер страницы в оглавлении не выводится
\pagenumbering{gobble}         %  номер страницы в оглавлении не выводится

\renewcommand\contentsname{Содержание}
\tableofcontents
%Поставьте строку \renewcommand{\contentsname}{Содержание} непосредственно перед командой \tableofcontents,

\cleardoublepage                    %  номер страницы в оглавлении не выводится
\pagenumbering{arabic}         %  номер страницы в оглавлении не выводится

\setcounter{page}{4}  % номер первой страницы после оглавления

\chapter*{ОБОЗНАЧЕНИЯ}
\addcontentsline{toc}{chapter}{ОБОЗНАЧЕНИЯ И СОКРАЩЕНИЯ}

В данной работе использованы следующие обозначения и сокращения.

\chapter*{ВВЕДЕНИЕ}
\addcontentsline{toc}{chapter}{ВВЕДЕНИЕ}

 Latex - самое качественное и удобное средство для верстки академических статей и монографий.

Рекомендуем начать работу с чтения Элементарного введения в \cite{__2003} - всего 43 страницы.

Затем стоит установить Latex на свой компьютер  в виде пакета Miktex, как описано в приложении \ref{latex}. 

Основным инструментом будет программа редактирования и трансляции Texworks. У нее единственное окно, в котором можно редактировать текст, и одна кнопка - зеленого цвета на панели сверху слева, которой можно запустить трансляцию текста. Исходный текст сохраняется, как файл с расширением .tex, в результате трансляции получается .pdf файл для просмотра и печати. Чтобы смотреть его, желательно установить бесплатный Adobe Acrobat Reader. В той папке, где у вас будут .tex и .pdf файлы, вы увидите другие рабочие файлы, на них можно не обращать внимание. Рекомендуем все файлы:  .tex, .pdf, файлы с рисунками и библиографическими записями держать в одной папке.

Ярлык программы Texworks можно скопировать из меню <<Все программы>> на рабочий стол.

Хорошо прочитать файл template.pdf, если будет интересно, как реализованы те или иные конструкции, параллельно можно открыть файл template.tex в Texworks.

Теперь вы готовы редактировать текст ВКР. Откройте template.tex //
в Texworks и сохраните под другим именем. Вносите изменения в текст шаблона мелкими порциями. После каждого небольшого изменения рекомендуем сохранять файл текста под другим именем, транслировать и смотреть, что получилось, так, если встретятся трудности, вы будете знать, что их вызвало. 

Учтите, что от начала шаблона до команды \verb|\begin{document}|  располагается преамбула с настройками, как правило, она останется неизменной до конца работы над ВКР. Собственно текст  располагается от команды \verb|\begin{document}| до команды \verb|\end{document}|.

Структурные элементы: реферат, введение, основная часть, заключение, приложения неизменны, они уже прописаны в шаблоне, например, так:\\
\newpage
\begin{verbatim}
\chapter*{ВВЕДЕНИЕ}
\addcontentsline{toc}{chapter}{ВВЕДЕНИЕ}
\end{verbatim}
Внутри Основной части вы используете команды \verb|\section| \\и \verb|\subsection|, чтобы создать разделы и подразделы.
В тексте вы используете команду \verb|\label|, чтобы пометить разделы, подразделы, рисунки, таблицы и формулы, и команду \verb|\ref|, чтобы на них сослаться.

Оглавление формируется командой \verb|\tableofcontents| автоматически, исходя из использованных в тексте команд структурирования.

Когда вы дойдете до списка использованной литературы, поставьте Zotero под Firefox, как описано в приложении \ref{zotero}. В Zotero вы сможете одним кликом сохранять библиографические ссылки из электронных каталогов. Затем вы экспортируете файл с расширением .bib и Latex автоматически сформирует список использованной литературы в соответствии с ГОСТ.

Отнеситесь внимательно к переходам от одного элемента к другому. Например, перед выводом оглавления стоят команды, отменяющие вывод номера страницы, а после - восстанавливающие этот режим.

Вносите изменения в текст постепенно и, возможно, вразброс, так как Latex автоматически нумерует разделы, рисунки, таблицы и формулы.

Сверстанная в Latex ВКР, возможно, будет выглядеть презентабельнее, чем по-другому. Кроме того более половины академических журналов принимают статьи, сверстанные в Latex, поэтому данный опыт поможет вам в будущем. 






%\chapter*{ОСНОВНАЯ ЧАСТЬ}
%\addcontentsline{toc}{chapter}{ОСНОВНАЯ ЧАСТЬ}

\chapter{Начало}

Создание выпускной работы в Latex - это подготовка текста, в котором встречаются команды Latex. Во многом в этом помогают предоставляемые шаблоны, например, готовая верстка титульной страницы, в которой нужно только заменить слова. Также помогают примеры, которые позволяют выполнять работу по аналогии или также, заменяя отдельные фрагменты. \\
Пример текста ( с командами курсива и полужирный):\\
%\begin{verbatim}
\verb|У попа {\it была} {\bf собака}|\\
%\end{verbatim}
У попа {\it была} {\bf собака}

\section{Структура текста}
Исходный файл для Latex состоит из {\it преамбулы} и {\it основного текста}. Преамбула, располагающаяся в начале текста, задает класс документа, используемые пакеты и прочие технические параметры \cite[][с.~10]{__2010}.

После преамбулы идет строка \verb|\begin{document}|, с которой и начинается основной текст. В самом конце файла нужно поместить строку\\
 \verb|\end{document}|\\
Пример (от знака \% идет комментарий):\\
\verb|\\documentclass[a4paper,14pt]{article}|\\
\% здесь преамбула\\
\verb|\begin{document}|\\
Привет мир\\
\verb|\end{document}|

\section{Символы}
Большинство символов в основном тексте обозначают сами себя, набирайте их прямо с клавиатуры.
Спецсимволы говорят Latex что-то специальное:\\
\verb|{   }   $   &   #   |\%\verb|   _   ^   ~   \|\\
Первые семь печатаются с помощью односимвольной команды, впереди добавляется  \verb|\|, например, \% печататеся командой \textbackslash\%.\\
Для \textbackslash{}   команда \verb|\textbackslash|.\\
Символы \verb|^| и \verb|~| можно напечатать, например, с помощью команды \verb|\verb|, смотрите раздел \ref{verb}.\\
Существует множество команд, которые печатают небуквенные символы, смотрите, например, \cite{__2003}.\\

\section{Команды Latex}
Некоторые команды требуют обязательные параметры, например,\\ \verb|\section{Название раздела}|.
У некоторых команд могут быть необязательные параметры, они заключаются в квадратные скобки:\\
\verb|\documentclass[a4paper,14pt]{article}|, здесь a4paper - размер бумаги, 14pt - нормальный размер шрифта 14.
Название команды Latex может быть одним небуквенным символом или состоять из нескольких букв. Во втором случае команда без параметров ограничивается справа пробелом, который при этом не печатается. Если это неудобно, то конец команды можно указать пустыми фигурными скобками \{\}.

Кроме команд в языке Latex используются {\it окружения} - фрагменты текста, заключенные между командами \verb|\begin{}| и \verb|\end{}|. Обе команды в качестве параметра принимают {\it имя окружения}. Окружение позволяет распространить свойство или действие на весь заключенный в окружение текст. Например, окружение large позволяет напечатать фрагмент текста увеличенным шрифтом. Для этого фрагмент заключается между \verb|\begin{large}| и \verb|\end{large}| \cite[][с.~14]{__2010}.

\section{Пробелы, строки и абзацы}
Любое количество идущих подряд {\it пробельных символов}, то есть пробелов, табуляций и переводов строки, Latex воспринимает как один пробел. Размер пробела между словами Latex выбирает сам, причем так, чтобы все пробелы в одной строке были одинаковы.

Однако пустая строка, то есть два подряд символа перевода строки, обозначает {\it разделитель абзацев}. Команда \verb|\noindent|, вставленная непосредственно перед абзацем, отменяет отступ (красную строку) для этого абзаца. Комбинация из двух символов  \textbackslash\textbackslash{} принудительно завершает строку, но не  абзац. Команда \verb|\par| означает то же, что пустая строка, то есть отделяет один абзац от другого.

Вставить между абзацами вертикальный интервал  можно, например, командами \verb|\bigskip|, \verb|\medskip| и \verb|\smallskip| \cite[][с.~14]{__2010} .

\section{Промежутки между словами и вертикальные отступы}

Некоторые слова нежелательно разрывать на разные строки, тогда между ними нужно ставить неразрывный пробел \verb|~|. Пример: на странице\verb|~|5.

Команда \verb|\quad| дает промежуток в 1em,  команда \verb|\qquad| - промежуток в 2em (em - промежуток, равный примерно ширине латинской буквы M текущего шрифта).

Для небольшого пробела используется команда \verb|\|,

Если необходимо задать промежуток с указанием конкретной длины, можно воспользоваться командой\\ 
\verb|\hspace{длина}|, например, \verb|\hspace{1.5cm}|  \cite[][с.~106]{__2003}. Если этот промежуток также должен сохраняться и в начале ( или конце~)  строки, то вместо команды \verb|\hspace| применяется \verb|\hspace*|.

Вертикальные отступы создаются командой \verb|\vskip|, например,\\ 
\verb|\vskip3.5cm|. 

\chapter{Управление шрифтом}\label{font}

\section{Команды переключения формы шрифта}

Форма шрифта характеризуется семейством (с засечками, без засечек, как у пишущей машинки или моноширинный), насыщенность (средняя, полужирная), начертание (прямое, курсивное, наклонное, capital). Имеются команды, в которых текст является параметром, внутри действия такой команды можно применять другие команды управления формой шрифта. Пример: \\
\textsf{собака \textbf{Барсик} и кошка \textbf{ Муся}}\\
\begin{minipage}{\textwidth}
\captionof{table}{Форма шрифта}
\begin{tabular}{|l|l|l|l|}
\hline
ru&en&команда&результат\\
\hline
с засечками&roman&\verb|\textrm{Слова}|&\textrm{Слова}\\
без засечек&sans serif&\verb|\textsf{Слова}|&\textsf{Слова}\\
пишущая машинка&typewriter&\verb|\texttt{Слова}|&\texttt{Слова}\\
средняя насыщенность&medium&\verb|\textmd{Слова}|&\textmd{Слова}\\
полужирная&boldface&\verb|\textbf{Слова}|&\textbf{Слова}\\
прямое начертание&upright shape&\verb|\textup{Слова}|&\textup{Слова}\\
курсивное&italic&\verb|\textit{Слова}|&\textit{Слова}\\
наклонное&slanted&\verb|\textsl{Слова}|&\textsl{Слова}\\
заглавные буквы&capital&\verb|\textsc{Слова}|&\textsc{Слова}\\
\hline
\end{tabular}
\end{minipage}

\section{Размеры шрифта}

Размер шрифта отсчитывается от нормального размера. В данном шаблоне нормальный размер, применяемый по умолчанию, 14pt. Из приводимых ниже команд ему соответствует \verb|\normalsize|. Другие команды по определенному алгоритму уменьшают или увеличивают размер. Эти команды могут применяться внутри группы текста, ограниченного фигурными скобками, тогда их действие за пределы скобок не распространяется. Пример :\\
\verb|{\small Путь} на {\large верх}|\\
{\small Путь} на {\large верх}.\\

\begin{table}[h!]
    \caption{Размеры шрифта}
    \centering
        \begin{tabular}{|c|c|}
        \hline  \verb|\tiny|      & \tiny        крошечный \\
        \hline  \verb|\scriptsize|   & \scriptsize  очень маленький\\
            \hline \verb|\footnotesize| & \footnotesize  довольно маленький \\
            \hline \verb|\small|        &  \small        маленький \\
            \hline \verb|\normalsize|   &  \normalsize  нормальный \\
            \hline \verb|\large|        &  \large       большой \\
            \hline \verb|\Large|        &  \Large       еще больше \\[5pt]
            \hline \verb|\LARGE|        &  \LARGE       очень большой \\[5pt]
            \hline \verb|\huge|         &  \huge        огромный \\[5pt]
            \hline \verb|\Huge|         &  \Huge        громадный \\ \hline
        \end{tabular}
\end{table}

\section{Расстояние между строками - интерлиньяж}

При смене размера шрифта ( непосредственно в том месте, где вставлена соответствующая команда ) определяется только размер символов. Что же касается расстояния между строками, то оно определяется в тот момент, когда очередная строка завершается ( либо по принудительной команде, либо, при верстке целого абзаца, по его окончании ). Возможна такая ситуация, когда вы начали писать абзац крупным шрифтом, а закончили мелким, что первые строки абзаца будут на очень маленьком расстоянии друг от друга. Самый простой способ, чтобы так не происходило, в одном абзаце использовать один размер.

\chapter[Рубрикация и оглавление]{Рубрикация\\ и оглавление}\label{toc}

\section{Команды рубрикации}
Для класса book, назначенного в этом шаблоне, внутри основной части можно использовать \verb|\chapter| - глава,
\verb|\section| - раздел, \verb|\subsection| - подраздел, \verb|\subsubsection| - подподраздел, \verb|\paragraph| - параграф, \verb|\subparagraph| - подпараграф. Эти команды имеют параметр - заголовок соответствующего раздела. Пример:\\
\verb|\section {Рубрикация и оглавление}|. Главы, разделы и подразделы автоматически нумеруются и помещаются в оглавление.

Если желательно избежать переноса слов в заголовке соответствующего раздела, то можно явно указать перенос на новую строчку. При этом можно разрешить перенос слов в оглавлении, если в квадратных скобках привести текст для оглавления. Пример:\\
\verb|\chapter[Рубрикация и оглавление]|\\\verb|{Рубрикация\\ и оглавление}|

\section{Оглавление}
Сформировать оглавление в Latex очень просто: достаточно в том месте, где вы хотите увидеть оглавление, вставить команду \\
\verb|\tableofcontents|

\section{Перекрестные ссылки}

Latex имеет команды, чтобы сослаться на раздел, страницу и другие части текста. Для этого место, на которое нужно сослаться, помечается командой \verb|\label|. В качестве параметра указывается метка, состоящая из латинских букв, цифр, символов двоеточия, подчеркивания и тире. Пример:\\
\verb|\section {Рубрикация и оглавление}\label{toc}|\\
...
Это было рассмотрено в разделе \verb|\ref{toc}| на стр.\verb|\pageref{toc}|\\
на печати:\\
Это было рассмотрено в разделе \ref{toc} на стр. \pageref{toc}\\

\section{Приложения}

Если в работе предусмотрены приложения, нужно поставить перед ними команду \verb|\appendix|. Эта команда дается только один раз, чтобы отделить текст всех приложений от основного текста. После команды \verb|\appendix| каждое приложение начинается с команды \verb|\chapter{}|, но вместо слова <<Глава>> выводится слово <<Приложение>>, а вместо номера арабскими цифрами выводится номер в виде заглавной латинской буквы.

\section{Ненумерованные заголовки}

Команда \verb|\section*{Заголовок}| напечатает заголовок в том же формате, что и обычные заголовки по команде \verb|\section{}|, но нумероваться он не будет. Такое же воздействие * оказывает на другие команды: \verb|\chapter*{}| и т.д. 

Ненумерованные заголовки автоматически не попадают в оглавление, нужно сделать так:\\
\verb|\chapter*{ОСНОВНАЯ ЧАСТЬ}|\\
\verb|\addcontentsline{toc}{chapter}{ОСНОВНАЯ ЧАСТЬ}|\\
команда \verb|\addcontentsline| поместит заданный текст в оглавление в формате главы.
Таким же образом в оглавление помещается заголовок над библиографией.

\chapter[Список литературы и ссылки на него]{Список литературы\\ и ссылки на него}

Ниже описано, как подготовить библиографию, чтобы Latex автоматически создал список использованной литературы в соответствии с ГОСТ.
\section{Накопить библиографию}
Установите Zotero под Firefox, как описано в приложении \ref{zotero}.
Большая буква Z справа вверху на панели Firefox свидетельствует о том, что Zotero установлено. Кликните мышкой на эту букву Z. В нижней половине экрана откроется Zotero.

Создадим новую коллекцию для вашей библиографии. Экран Zotero обычно делится на 3 колонки. В левой колонке вверху подведите правую мышку к заголовку <<Моя библиотека>>. В появившемся меню выберите <<Новая подборка>> и введите название, например, <<bibliogr>>. 

Заполним созданную подборку необходимой библиографией. Откроем в Firefox Scopus, Web of science, РИНЦ или электронный каталог Российской национальной библиотеки.\\
В нижней половине экрана у вас открыто Zotero. Подборка, где будет собираться библиография, выделена мышкой.\\
Проведите поиск. Справа от большой буквы Z справа вверху на панели Firefox появится иконка: желтая, если найдено несколько документов, или для одного документа - с маленьким изображением титульного листа. Нажмите на эту иконку. Если потребуется, отметьте мышкой нужные документы и подтвердите. В средней колонке Zotero начнут появляться документы. Выделите мышкой один из них. В правой колонке в закладке <<Информация>> вы увидите значения всех библиографических полей.

Созданные давно статьи и монографии могут отсутствовать в современных каталогах. Кроме того каталоги иногда содержат ошибки. В обоих случаях необходимо уметь вводить и редактировать библиографическую запись вручную. Чтобы отредактировать запись, выделите в средней колонке документ. В правой колонке в закладке <<Информация>> появятся поля этой записи. Любое из них можно редактировать. По поводу автора: из выпадающего списка можно выбрать роль - редактор, автор и т.д., имеется два подполя - для фамилии и инициалов, для второго и т.д.  автора - нажать на + справа.

Для ввода вручную нового документа нужно нажать мышкой зеленый кружок с белым плюсиком над средней колонкой $\vcenter{\hbox{\includegraphics{image01.jpg}}}$, выбрать тип документа - книга, статья из журнала и т.д., и отредактировать поля в правой колонке.  
 
 \section{Файл с библиографией}
 
 Подготовим файл с библиографией в формате biblatex и с расширением~.bib.
 
 Откроем Firefox,  откроем Zotero в нижней половине экрана, нажав на большую Z,  в левой колонке найдем подборку, в которой накоплена библиография для выпускной работы, и подведем к ней правую мышку.\\
 Выберем <<Экспортировать подборку>>.\\
В выпадающем списке выберем Формат = Biblatex. Нужно убрать, если есть, отметки в строчках Экспорт заметок, Экспорт файлов, Использовать сокращенное название журнала. И жмем ОК. Сохраняем файл с именем из латинских букв в папку, где верстаем выпускную работу.

\section{Как формируется список}

В преамбуле вызывается пакет biblatex.\\
В команде \verb|\addbibresource{bibliogr.bib}|  Latex получает информацию, что библиография находится в файле bibliogr.bib.\\
В конце текста, но перед приложениями команда\\
\verb| \printbibliography|\\
\verb|[title=Список \\использованных источников]|\\
печатает список библиографии.\\
Команда\\
\verb|\addcontentsline{toc}{chapter}|\\
\verb|{Список использованных источников}|\\
включает список в оглавление.\\
Команда \verb|\nocite{*}| говорит о том, что нужно выдать в списке использованной литературы не только документы, на которые есть ссылки из текста выпускной работы, но все содержимое файла с библиографическими записями. По окончании отладки текста эту команду лучше убрать.


Все это присутствует в шаблоне и вам остается аккуратно использовать шаблон в верстке выпускной работы. Нужно только в команде\\  
\verb|\addbibresource{bibliogr.bib}| заменить имя файла на используемое вами, когда вы экспортировали библиографию из Zotero.

\section{Ссылки}

В тексте можно сослаться на документ из списка использованной литературы с помощью метки библиографической записи. Метки формируются Zotero автоматически, они находятся в первой строке описания библиографической записи. Например, в файле bibliogr.bib, прилагаемом к данному шаблону, монография А.В.Столярова снабжена меткой \verb|__2010|. Сослаться на эту монографию можно, употребив в тексте команду \verb| \cite{__2010}|. В квадратных скобках появится номер документа, как он напечатан в списке использованной литературы. Если нужно сослаться на страницу 10, то команда будет такой: \verb|\cite[][с.~10]{__2010}|. Текст из вторых квадратных скобок появится после номера документа. Если нужно поместить какой-то текст до номера документа, то его нужно разместить  внутри первых кадратных скобок, иначе эти скобки пустые.

\section{Англоязычные документы}

Англоязычные документы в списке использованной литературы представляются библиографической записью на английском языке. В этом случае необходимо стандартные слова <<страницы>>,<< том>>, <<редактор>> и т.д. представить на английском.

В данном шаблоне Latex по умолчанию считает библиографическую запись составленной по-русски. Если есть записи на английском языке, то нужно отредактировать файл .bib с библиографией. В такие записи нужно вставить поле \\
\verb|hyphenation={english}|. \\
Смотреть и редактировать файлы с расширением .bib удобно в бесплатном редакторе notepad++. Где его скачать легко найти в GOOGLE.

\chapter[Списки, сноски, дословная печать]{Списки, сноски, \\гиперссылки, \\дословная печать}

\section{Списки}

Иногда возникает необходимость перечислить объекты, методы и т.д. Списки можно вкладывать друг в друга.

\subsection{Простые перечисления}

Зарубежные новостные агенства:\\
\verb|\begin{itemize}|\\
\verb|\item REUTERS|\\
\verb|\item CNN|\\
\verb|\item BBC|\\
\verb|\end{itemize}|\\

на печати:

Зарубежные новостные агенства:
\begin{itemize}
\item REUTERS
\item CNN
\item BBC
\end{itemize}

\subsection{Перечисления с нумерацией}

Стихотворные размеры:\\
\verb|\begin{enumerate}|\\
\verb|\item Хорей|\\
\verb|\item Ямб|\\
\verb|\end{enumerate}|\\

\begin{minipage}{\textwidth}
на печати:

Стихотворные размеры:
\begin{enumerate}
\item Хорей
\item Ямб
\end{enumerate}
\end{minipage}

\subsection{Перечни без пустот}

( работает в данном шаблоне )

\smallskip
Постановка научной проблемы:\\
\verb|\begin{compactlist}|\\
\verb|\item обнаружение недостаточности знаний, противоречий|\\
\verb|\item осознание потребности|\\
\verb|\item формулирование проблемы|\\
\verb|\end{compactlist}|

\medskip
на печати:

Постановка научной проблемы:
\begin{compactlist}
\item обнаружение недостаточности знаний, противоречий
\item осознание потребности
\item формулирование проблемы
\end{compactlist}

\section{Сноски}

Сноски оформляются с помощью команды \verb|\footnote{<текст>}|. В том месте, где команда встретилась, появляется номер сноски, набранный цифрами уменьшенного размера и сдвинутый вверх. Текст сноски, снабженный тем же номером, размещается в нижней части страницы. От основного текста сноски отделяются горизонтальной чертой \cite[][с.~58]{__2010}. 

В рамках каждой главы сноски имеют сквозную нумерацию; при смене главы ( то есть по команде \verb|\chapter| ) счетчик сносок сбрасывается, так что первая сноска каждой главы имеет номер 1.

Иногда возникает желание сослаться на одну и ту же сноску несколько раз. Для этого первое вхождение сноски следует оформить как обычно и снабдить меткой, например, так:\\ 
\verb|\label{my_footnote}|. \\
А в остальных местах применить команду\\
\verb|\footnotemark[\ref{my_footnote}]|.

\section{Гиперссылки}

www-адрес, как ссылка: \verb|\url{http://www.rsl.ru/}|\\
\url{http://www.rsl.ru/}

Сайт: \verb|\href{http://urfu.ru/}{УРФУ}|\\
Сайт: \href{http://urfu.ru/}{УРФУ}.


\section{Воспроизведение <<как есть>>}\label{verb}

В тексте или специальных листингах, например, в программах могут\\встречаться слова, которые воспринимаются Latex, как команды. Для этих случаев есть инструмент, который воспроизводит текст <<как есть>>.

Начиная с команды \verb|\begin{verbatim}| и до тех пор, пока не встретится строка \verb|\end{verbatim}| Latex все встреченные символы будет выдавать моноширинным шрифтом, причем никакие символы ( кроме комбинации 
\verb|\end{verbatim}|  ) не будут интерпретироваться никаким специальным образом.

Для  фрагментов внутри строки применяется команда $\backslash$verb. Выдаваемый текст с обоих сторон ограничивается одним и тем же символом, например, вертикальной чертой, и прижимается вплотную к  $\backslash$verb справа без пробела.

\chapter{Таблицы}

\section{Описание таблицы}

Таблица создается окружением tabular. Пример:\\


\begin{tabular}{|c|c|}
\hline
мужские имена&женские имена\\
\hline
Александр&Анастасия\\
Роман&Рената\\
\hline
\end{tabular}

\begin{verbatim}
\begin{tabular}{|c|c|}
\hline
мужские имена&женские имена\\
\hline
Александр&Анастасия\\
Роман&Рената\\
\hline
\end{tabular}
\end{verbatim}

В строке \verb|\begin{tabular}| количество букв <<c>> - это количество столбцов формируемой таблицы.\\
Буква <<c>> говорит о выравнивании по центру в соответствующем ей по порядку расположения столбце. Другие возможные буквы вместо c: l - выравнивание влево, r - выравнивание вправо. \\
Вертикальные черточки по краям и между буквами порождают вертикальные линии на всю высоту таблицы.

\verb|\hline| порождает горизонтальную линию между строками таблицы.\\Строки таблицы завершаются командой \verb|\\|. Между ячейками одной строки ставится \&. Количество ячеек в каждой строке должно соответствовать количеству столбцов, заданному в строке  \verb|\begin{tabular}|.

\section{Дополнительные возможности}

Пример с ячейкой на несколько столбцов \cite[][с.~47]{__2010}:


\begin{tabular}{|r|c|p{0.15\textwidth}|
p{0.25\textwidth}|c|r|}
\hline 
№ & Тип & Автор & Заглавие & Год & Тир. \\ 
\hline
1 & книга & Артур Конан Дойл & Собака Баскервилей
& 1975 & 10\,000 \\ \hline
2 & книга & Жюль Верн & Пять недель на воздушном шаре
& 1981 & 7000 \\ \hline
3 & журнал & \multicolumn{2}{c|}{Вокруг света (№ 5)}
& 1995 & 5000 \\ \hline
\end{tabular}

\begin{verbatim}
\begin{tabular}{|r|c|p{0.15\textwidth}|
p{0.25\textwidth}|c|r|}
\hline 
№ & Тип & Автор & Заглавие & Год & Тир. \\ 
\hline
1 & книга & Артур Конан Дойл & Собака Баскервилей
& 1975 & 10\,000 \\ 
\hline
2 & книга & Жюль Верн & Пять недель на воздушном шаре
& 1981 & 7000 \\ 
\hline
3 & журнал & \multicolumn{2}{c|}{Вокруг света (№ 5)}
& 1995 & 5000 \\ 
\hline
\end{tabular}
\end{verbatim}

Ячейка на несколько столбцов организуется командой \verb|\multicolumn|.\\
Первый параметр - количество столбцов, на которые распространяется ячейка.\\
Второй параметр - буква (c,l или r), задающая выравнивание в этой ячейке, кроме того, может потребоваться обозначить  вертикальными черточками вертикальные границы ячеки - одну или обе.\\
Третий параметр - текст, помещаемый в ячейку.

В этом же примере есть столбцы, в которых текст в ячейках занимает несколько строк. Для этих столбцов вместо буквы, задающей выравнование, нужно указать размер ширины столбца в формате \verb|p{ширина}|. Примеры:\\
\verb|p{3.5cm}| - 3,5 сантиметра,\\
\verb|p{50mm}| - 50 миллиметров,\\
\verb|p{0.25\textwidth}| - четверть от ширины документа.

\section{Плавающие объекты}\label{plav}

Когда в документ требуется вставить рисунок или таблицу, обычно в тексте ставится ссылка ( что-то вроде <<см. рис. 2>> или <<см. табл. 8>> ), а сами таблицы или рисунки размещаются вверху или внизу текущей страницы. Это вызвано тем, что жесткая привязанность к тексту приводит к тому, что остаются незаполненные пустые места.

Latex позволяет описывать объекты, которые следует поместить вблизи места, где вставлено их описание, но конкретное размещение можно выбрать, исходя из удобства и эстетичности. Такие объеты называются плавающими \cite[][с.~43]{__2010}. 

Описание плавающей таблицы \ref{tab}:

\begin{verbatim}
\begin{table}[htbp]
\caption{Пример}\label{tab}
\begin{tabular}{cc}
A&B\\
C&D\\
\end{tabular}
\end{table}
\end{verbatim}

\begin{table}[htbp]
\caption{Пример}\label{tab}
\begin{tabular}{cc}
A&B\\
C&D\\
\end{tabular}
\end{table}

Окружение\\
\verb|\begin{table}[htbp]|\\
\ldots\\
\verb|end{table}|\\
сообщает Latex, что  внутри него находится описание плавающей таблицы. В квадратных скобках можно дать рекомендации по размещению в порядке убывания желательности:\\
h - (here) сразу после появления в исходном тексте,\\
t - (top) вверху страницы,\\
b - (bottom) внизу страницы,\\
p - (page) на отдельной странице.\\ 
Пожелание можно превратить в требование, добавив восклицательный знак. Latex в этом случае сделает все, что сможет, но требуемое размещение не гарантирует.

Команда \verb|\caption{Пример}\label{tab}|  обеспечивает заголовок\\таблицы, нумерацию и вывод номера. Если caption пометить \\ 
(командой \verb|\label{tab}|), то на таблицу можно сослаться с помощью использованной метки, например, так:\\
\ldots данные размещены в таблице \verb|\ref{tab}| \\
\ldots данные размещены в таблице \ref{tab}. 

Можно использовать не плавающие объекты вперемежку с плавающими, при этом нумерация будет правильной.

\section{Не плавающие таблицы}

Описание не плавающей таблицы \ref{tabex}:

\begin{verbatim}
{
\captionof{table}{Пример}\label{tabex}
\begin{tabular}{cc}
A&B\\
C&D\\
\end{tabular}
}
\end{verbatim}

Обратите внимание, весь комплект одет в фигурные скобки.
Если надпись над таблицей и сама таблица оказались на разных страницах, то весь комплект оденьте в окружение\\
\verb|\begin{minipage}{\textwidth}|\\
\verb|\end{minipage}|.

{
\captionof{table}{Пример}\label{tabex}
\begin{tabular}{cc}
A&B\\
C&D\\
\end{tabular}
}

Команда \verb|\captionof{table}{Пример}| обеспечивает заголовок таблицы, нумерацию и вывод номера. Если captionof пометить командой \\ 
\verb|\label{tabex}|, то на таблицу можно сослаться с помощью использованной метки, например, так:\\
\ldots данные размещены в таблице \verb|\ref{tabex}| \\
\ldots данные размещены в таблице \ref{tabex}. 

\chapter{Рисунки}

\section{Описание рисунка}

Рисунки из файлов вставляются командой \verb|\includegraphics|.\\
Примеры.\\
\verb|\includegraphics[width=0.9\textwidth]{krug.jpeg}| - рисунок займет 90\% полосы набора;\\
\verb|\includegraphics[scale=0.75]{clinical.pdf}| - размеры рисунка умножатся на коэффициент 0.75.\\
Расширение файла можно опускать.\\
Допускаюся векторные форматы .pdf и .eps, растровые форматы .jpg и .png. Векторные форматы предпочтительнее.\\
Рисунки можно собрать в отдельную папку и указать в команде абсолютный или относительный путь к файлу, например, так:\\
\verb|\includegraphics{images/train.eps}|\\

\section{Плавающие рисунки}

Рисунок становится плавающим с помощью окружения \verb|\begin{figure}|. 

Пример описания рисунка \ref{fig_clinics}:
\begin{verbatim}
\begin{figure}[htbp]
\includegraphics[width=0.7\textwidth]{zclinical.pdf}
\caption{Приборы для медицины}
\label{fig_clinics}
\end{figure}
\end{verbatim}

\begin{figure}[htbp]
\includegraphics[width=0.7\textwidth]{zclinical.pdf}
\caption{Приборы для медицины} 
\label{fig_clinics}
\end{figure}

Поведение плавающих объектов и опции [htbp] - описаны в разделе \ref{plav}. Подпись под рисунком, нумерация и вывод номера обеспечивается командой \verb|\caption|. \\
Плавающие рисунки могут использоваться вперемежку с неплавающими, при этом нумерация будет общая.\\
На рисунок можно сослаться с помощью метки, использованной в команде \verb|\label|.\\
\ldots измерения проведены на приборе \ldots см. рис. \verb|\ref{fig_clinics}|\\
\ldots измерения проведены на приборе \ldots см. рис. \ref{fig_clinics}

\section{Не плавающие рисунки}

Пример описания рисунка \ref{fig_krug}:\\
\begin{verbatim}
{
\includegraphics{zkrug.jpg}
\captionof{figure}{Соотношение составляющих}
\label{fig_krug}
}
\end{verbatim}

\begin{minipage}{\textwidth}
{
\includegraphics{zkrug.jpg}
\captionof{figure}{Соотношение составляющих}\label{fig_krug}
}
\end{minipage}

Обратите внимание, что весь комплект одет в фигурные скобки.
Если рисунок и подпись под ним оказались на разных страницах, то весь комплект оденьте в окружение\\
\verb|\begin{minipage}{\textwidth}|\\
\verb|\end{minipage}|.

Команда \verb|\captionof{figure}|  обеспечивает подпись под рисунком, нумерацию и вывод номера на печать.\\
Команда \verb|\label{fig_krug}| вставляет метку, благодаря чему можно сослаться на рисунок:\\
\ldots смотрите рисунок \verb|\ref{fig_krug}|\\
\ldots смотрите рисунок \ref{fig_krug}

\chapter{Формулы}

Верстка формул блестяще описана в   \cite{__2003} и \cite{__2010}. Если ваша статья содержит формулы, рекомендуем ознакомиться с этими материалами. \\
Пример описания формулы \ref{eq}:

\begin{verbatim}
\begin{equation}
\label{eq}
\sum_{i=1}^n n^2=\frac{n(n+1)(2n+1)}{6}
\end{equation}
\end{verbatim}

\begin{equation}
\label{eq}
\sum_{i=1}^n n^2=\frac{n(n+1)(2n+1)}{6}
\end{equation}

\chapter*{ЗАКЛЮЧЕНИЕ}
\addcontentsline{toc}{chapter}{ЗАКЛЮЧЕНИЕ}

Не сомневайтесь в своих силах. На самом деле Latex - самое качественное и удобное средство для верстки академических статей и монографий.

Шаблон для верстки ВКР магистра лежит на сайте \href{http://vkr.urfu.ru/}{vkr.urfu.ru} в разделе документы.

Вопросы по верстке пишите в \href{http://lib.urfu.ru/course/view.php?id=83}{виртуальную справку} на сайте Зональной библиотеки.

\href{http://lib.urfu.ru/file.php/154/LaTeX-book.pdf}{Книгу С.М. Львовского} можно скачать с сайта УРФУ. Книга А.В. Столярова выложена на \href{http://www.stolyarov.info/books/latex3days}{сайте автора}.

Авторы данного шаблона выражают благодарность Данилу Александровичу Федоровых, старшему преподавателю Высшей школы экономики\\(Москва). Мы взяли за основу его настройки Latex в виде преамбулы, дополнив в соответствии с требованиями УРФУ.

Благодарим Сергея Михайловича Львовского и Андрея Викторовича Столярова, авторов замечательных книг по Latex.

Благодарим сообщество профессионалов Latex \href{tex.stackexchange.com}{tex.stackexchange.com} и Олега Доманова, автора пакета Biblatex-GOST.

%\addcontentsline{toc}{section}{Список литературы}
%%%%%%%%%%%%%%%%%%%%%%%%%%%%%
\nocite{*}
\renewcommand{\bibname}{Список использованных источников}
\printbibliography[title=Список \\использованных источников] 
\addcontentsline{toc}{chapter}{СПИСОК ИСПОЛЬЗОВАННЫХ ИСТОЧНИКОВ}
%%%%%%%%%%%%%%%%%%%%%%%%%%%%%%%%%%%%%
\begin{thebibliography}{3}
\bibitem{Sulsky1994}
Sulsky D., Chen Z., Schreyer H. L.  A particle method for history-dependent materials // Computer Methods in Applied Mechanics and Engineering. --- 1994, V. 118. --- P. 179--196.
\bibitem{LiuLiu}
Liu G. R., Liu M. B. Smoothed particle hydrodynamics: a meshfree particle method. --- Singapore : World Scientific Publishing. --- 2003. --- 449 p.
\end{thebibliography}
%%%%%%%%%%%%%%%%%%%%%%%%%%%%%%%%%%%%%%%%%
\appendix

\chapter*{ПРИЛОЖЕНИЯ}
\addcontentsline{toc}{chapter}{ПРИЛОЖЕНИЯ}

\chapter{Установка Latex}\label{latex}

Откройте \href{http://miktex.org/}{http://miktex.org/}, в верхнем меню нажмите Download.

В разделе Other downloads скачайте Miktex Net Installer для Windows 32-bit или Windows 64-bit в зависимости от установленной на компьютере операционной системы. Эти инсталляторы помечены текстом:\\
This installer allows you download all packages and install a complete TeX$\backslash$ LaTeX system.

Запустите инсталлятор от пользователя с правами администратора.

Архивы результатов такого скачивания некоторой давности лежат на сайте \href{http://vkr.urfu.ru/}{vkr.urfu.ru} в разделе Документы.

Среди скачанных файлов найдите и запустите (с правами администратора) файл setup.....exe.

По окончании  проверьте установку. Запустите редактор/транслятор\\Texworks, входящий в состав пакета Miktex, и при всех параметрах по умолчанию исполните пример:\\

\begin{verbatim}
\documentclass[12pt]{article}   
\usepackage{amsmath}
\begin{document}
$$\begin{pmatrix}
a_{11}-\lambda & a_{12}&a_{13}\\
a_{21}& a_{22}-\lambda &a_{23}\\
a_{31}& a_{32}&a_{33}-\lambda
\end{pmatrix}$$
\end{document}
\end{verbatim}

$$\begin{pmatrix}
a_{11}-\lambda & a_{12}&a_{13}\\
a_{21}& a_{22}-\lambda &a_{23}\\
a_{31}& a_{32}&a_{33}-\lambda
\end{pmatrix}$$
\vskip1cm
Теперь нужно настроить Miktex на определенный режим работы:\\
- среди множества вариантов трансляторов, заложенных в Miktex, будем использовать Xelatex;\\
- библиографию из файла с расширением .bib будет забирать программа Biber;\\
- оформлять список использованной литературы в соответствии с ГОСТ будет пакет biblatex.

Создайте .bat файл с именем xelatex+biber.bat и поместите его, считая от корневого каталога Miktex, в  \verb|\miktex\bin\x64| - для Miktex 64 и в \verb|\miktex\bin\| - для Miktex 32. Содержимое файла:\\

\begin{verbatim}
miktex-xetex.exe -synctex=1 -undump=xelatex "%1"
biber.exe "%2"
miktex-xetex.exe -synctex=1 -undump=xelatex "%1"
\end{verbatim}

Запустите Texworks.\\
Edit - preferentces - typesetting\\
и в разделе Processingtools кнопкой + создайте:\\
Name:\\
xelatex+biber\\
Program:\\
xelatex+biber.bat\\
и 2 аргумента в указанном порядке:\\
\$fullname\\
\$basename\\
и отметьте: view PDF after running

нажмите ОК.

Ниже раздела Processingtools в выпадающем списке, помеченном, ка\\Default, выберите xelatex+biber и сохраните клавишей OK. Теперь на верхней панели Texworks правее зеленой кнопки стоит xelatex+biber.\\
Попробуйте оттранслировать данный  шаблон (файл с расширением .tex), если получился текст, который вы сейчас читаете, настройка прошла успешно.

\chapter{Установка Zotero}\label{zotero}

В Firefox открываем  zotero.org,  жмем в верхнем меню Download, жмем Zotero for Firefox, разрешаем и подтверждаем установку расширения. После перезапуска Firefox на его панели справа вверху должна появиться большая буква Z. 

\end{document} % конец документа
