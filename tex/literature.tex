\begin{thebibliography}{100}
%список источников
%\bibitem{Кнорринг1976}
%Кнорринг, Г.М. Справочная книга для проектирования электрического освещения.~/
%Г. М. Кнорринг, Ю. Б. Оболенцев, Р. И. Берим, В. М. Крючков; Под ред. Г.Б. Кнорринга. Л.: Энергия, 1976. 384с.
\bibitem {Мэттьюс1897}
Mathews, G. B. On the partition of numbers ~/
G.B. Mathews~/~/
Proceedings of the London Mathematical Society. 28: С. 486–490. 
\bibitem{Петерсен1967}
C.C.Petersen "Computational experience ~/
with variants of the Balas algorithm applied to the selection
of R\&D projects" Management Science 13(9) (1967) 736-750.
\bibitem{Фразер1970}
Fraser Alex. Computer Models in Genetics. — New York: McGraw-Hill, 1970. — ISBN 0-07-021904-4.
\bibitem{Чу1998}
P.C.Chu and J.E.Beasley "A genetic algorithm for the multidimensional knapsack problem"~/
 Journal of Heuristics, vol. 4, 1998, С. 63-86.
\bibitem{Гэвиш1982}
Gavish, B. and H. Pirkul. (1982). “Allocation of Databases and Processors in a Distributed Computing
System.” In J. Akoka (ed.) Management of Distributed Data Processing, North-Holland, с. 215–231.
\bibitem{Ших1979}
Shih,W. (1979). “A Branch and Bound Method for the Multiconstraint Zero-One Knapsack Problem,” Journal of
the Operational Research Society 30, 369–378.
\bibitem{Гилмор1966}
Gilmore, P.C. and R.E. Gomory. (1966). “The Theory and Computation of Knapsack Functions,” Operations
Research 14, 1045–1075.
\end{thebibliography}
