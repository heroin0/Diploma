\chapter{Теоретическая часть}

\section{Задача о многомерном рюкзаке}
В данной работе рассматривается задача о многомерном рюкзаке(Multidimensional 0-1 knapsack problem, MKP).
Эта задача является модификацией классической задачи о рюкзаке, поставленной в 19 веке Джорджем Мэттьюсоном. (см \cite{Мэттьюс1897})
Данный же вариант задачи впервые был предложен Клиффордом Петерсеном в 1967 году.(см \cite{Петерсен1967})
\\Постановка задач такова

%Неформальный вариант 
Пусть существует N предметов, каждый из которых имеет стоимость $c_i$ и размеры $s_{ij}$, где $i\in{1,2,...,N}$,$j\in{1,2,...,M}.$
Пусть также существует рюкзак с ограничениями по вместимости по измерениям $r_j$. 
Требуется максимизировать сумму
\[\sum_{i=1}^N{c_i x_i}\]
где $x_i\in\{0,1\}$ при условии
\[\sum_{i=1}^N{s_{ij} x_i}< r_j\]
 $\forall j\in\{1,2,…,M\}$


И стандартная задача, и её модификация являются NP-полными задачами. 
Решение задач такого рода довольно сложно, бла-бла.
Вычислительная сложность задачи такого рода при переборном решении для N предметов - \[о(2^N)]\, что, вкупе с NP-сложностью, делает алгоритмическое решение такой задачи неэффективным для больших N.

\section{Машинное обучение и генетические алгоритмы}

Генетические алгоритмы являются семейством в множестве эвристических алгоритмов. Впервые такой алгоритм был предложен А. Фразером (см \cite{Фразер1970})
Эвристические алгоритмы – алгоритмы, для которых их корректность строго не доказана, либо наоборот, доказана их некорректность в некоторых случаях. 
Генетический алгоритм моделирует естественные процессы эволюции популяции.
Решение задачи с помощью такого алгоритма требует нескольких предварительных этапов:
	Выбор кодирования генотипа.
На этом этапе нужно выбрать способ кодирования генотипа, который будет удобен для данной задачи. Такой генотип однозначно определяет фенотип, однозначно моделируя сущность, рассматриваемую в задаче.
	Выбор мутации.
Генотипы подвергаются каким-либо мутациям, то есть каким-то образом изменяются их составляющие.
	Выбор механизма скрещивания (кроссинговера).
На данном этапе происходит создание новых генотипов из частей старых. 
	Выбор функции оценки(фитнесс-функции)
Такая функция позволяет оценивать генотипы и отбирать из них лучшие.
