\chapter{Теоретическая часть}

\section{Задача о многомерном рюкзаке}
В данной работе рассматривается задача о многомерном рюкзаке\\(Multidimensional 0-1 knapsack problem, MKP).
Эта задача является модификацией классической задачи о рюкзаке, поставленной в 19 веке Джорджем Мэттьюсоном. (см \cite{Мэттьюс1897})
Данный же вариант задачи впервые был предложен Клиффордом Петерсеном в 1967 году.(см \cite{Петерсен1967})
\\Постановка задач такова

%Неформальный вариант 
Пусть существует N предметов, каждый из которых имеет стоимость $c_i$ и размеры $s_{ij}$, где $i\in{1,2,...,N}$,$j\in{1,2,...,M}.$
Пусть также существует рюкзак с ограничениями по вместимости по измерениям $r_j$. 
Требуется максимизировать сумму
\[\sum_{i=1}^N{c_i x_i}\]
где $x_i\in\{0,1\}$ при условии
\begin{equation}\label{Valid}
\sum_{i=1}^N{s_{ij} x_i}< r_j
 \forall j\in\{1,2,…,M\}\
\end{equation}
 
И стандартная задача, и её модификация являются NP-полными задачами. 
%можно доказать
Вычислительная сложность задачи такого рода при переборном решении для N предметов - \[o(2^N)\], что, вкупе с NP-сложностью, делает алгоритмическое решение такой задачи неэффективным для больших N.
Однако такие задачи могут быть решены эвристическими алгоритмами, то есть алгоритмами, для которых их корректность строго не доказана. 

\section{Генетические алгоритмы}

Генетические алгоритмы являются семейством в множестве эвристических алгоритмов. Впервые такой алгоритм был предложен А. Фразером. (см \cite{Фразер1970})
%эффективно применяются для решения одномерных рюкзаков - найти пруфы.
Алгоритм является итеративным.
Генетический алгоритм моделирует естественные процессы эволюции популяции, а именно - мутацию и скрещивание.
Решение задачи с помощью такого алгоритма требует нескольких предварительных этапов:
\begin{itemize}
	\item Выбор кодирования генотипа.\\
На этом этапе нужно выбрать способ кодирования генотипа, который будет эффективен для данной задачи. Такой генотип должен однозначно моделировать сущность, рассматриваемую в задаче.
	\item Выбор начального приближения.\\
Для запуска итерационного процесса требуется создать начальное множество - пул генотипов.
	\item Выбор мутации.\\
На каждой итерации алгоритма часть пула генотипов подвергется мутациям, то есть определенным образом изменяются их составляющие.
	\item Выбор механизма скрещивания (кроссинговера).\\
После мутации происходит создание новых генотипов из частей старых с сохранением признаков родителя. Алгоритм скрещивания позволяет получить из двух родительских генотипов два различных дочерних генотипа.
	\item Выбор функции оценки(фитнесс-функции).\\
Такая функция позволяет оценивать генотипы с точки зрения их близости к оптимальному решению и отбирать из них лучшие на каждой итерации.
\end{itemize}
\section{Этапы работы алгоритма}%Тоже переделать.
\begin{itemize}%Это место стоит проверки
\item Создается пул генотипов с импользованием заданного алгоритма начального приближения
\item Запускается итерационный процесс
	\subitem Случайным образом выбирается часть пула, которая подвергнется мутации
	\subitem Выбранная часть пула генотипов мутируется с использованием заданного алгоритма мутации
	\subitem Мутировавшие генотипы замещают собой исходные в пуле, немутировавшие остаются без изменений
	\subitem Из пула генотипов выбираются пары для скрещивания
	\subitem Производится скрещивание с использованием заданного алгоритма
	\subitem С использованием заданной функции оценки из результатов скрещивания выбираются лучшие 
	\subitem Если выполнено условие останова - например, достигнут предел числа итераций или известный максимум, то итерационный процесс завершается, в противном случае  начинается следующая итерация.
\item Результат итерационного процесса отдается пользователю
\end{itemize}
\section{Выбор этапов}%ПЕРЕДЕЛАЙ И ПРОВЕРЬ
Наиболее естественным кодированием отдельного решения задачи о рюкзаке в генотип является бинарная последовательность длины N, состоящая из нулей и единиц.
Каждый i-й элемент такой последовательности является индикатором вхождения i-го предмета в текущее решение. Такая модель требует наличия проверки коееректности генотипа - соблюдения условия \ref{Valid} 
\\ Для генерации начального приближения был использован жадный алгоритм. Сначала создается генотип из единиц, соответствующий конфигурации рюкзака, в который положены все предметы. Затем в случайном порядке единицы заменяются на нули, пока полученныая конфигурация не будет удовлетворять условию коррекности. После этого полученный генотип мутируется с помощью текущей мутации до заполнения пула решений.
\\ В ходе работы было реализовано несколько алгоритмов мутации и скрещивания с целью сравнения их эффективности. Были реализованы следующие алгоритмы мутации:
\begin{itemize}
	\item Мутация в одной позциции, при которой заменяется значение в одной случайно выбранной точке генотипа.
	\item Инверсионная мутация, при которой половина генотипа заменяется на противоположные значения.
\end{itemize}
 Были реализованы следующие алгоритмы скрещивания:
 \begin{itemize}
	\item Скрещивание по 1 точке, при котором выбирается произвольная точка в последовательнсти генотипа, значения до точки берутся от первого генотипа, после - от второго. 
	\item Скрещивание по двум точкам, при котором выбираются две различные произвльные точки, значения внутри интервала и в самих точках берутся из первого генотипа, вне интервала - из второго. 
	\item Побитовае скрещивание, при котором значения на нечетных позициях берутся из первого генотипа, на четных - из второго.  
 \end{itemize}
 В качестве функции оценки используется стоимость всех предметов, содеражщихся в рюкзаке, соответствующем конфигурации.
 \\
%Борьба с локальными максимумами