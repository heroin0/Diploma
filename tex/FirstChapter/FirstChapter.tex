\chapter{Теоретическая часть}

\section{Задача о многомерном рюкзаке}
В данной работе рассматривается задача о многомерном рюкзаке\\(Multidimensional 0-1 knapsack problem, MKP).
Эта задача является модификацией классической задачи о рюкзаке, поставленной в 19 веке Джорджем Мэттьюсоном. (см \cite{Мэттьюс1897})
Данный же вариант задачи впервые был предложен Клиффордом Петерсеном в 1967 году.(см \cite{Петерсен1967})
\\Постановка задач такова

%Неформальный вариант 
Пусть существует N предметов, каждый из которых имеет стоимость $c_i$ и размеры $s_{ij}$, где $i\in{1,2,...,N}$,$j\in{1,2,...,M}.$
Пусть также существует рюкзак с ограничениями по вместимости по измерениям $r_j$. 
Требуется максимизировать сумму
\[\sum_{i=1}^N{c_i x_i}\]
где $x_i\in\{0,1\}$ при условии
\begin{equation}\label{Valid}
\sum_{i=1}^N{s_{ij} x_i}< r_j
 \forall j\in\{1,2,…,M\}\
\end{equation}
 
И стандартная задача, и её модификация являются NP-полными задачами. 
%можно доказать
Вычислительная сложность задачи такого рода при переборном решении для N предметов - \[o(2^N)\], что, вкупе с NP-сложностью, делает алгоритмическое решение такой задачи неэффективным для больших N.
Однако такие задачи могут быть решены эвристическими алгоритмами, то есть алгоритмами, для которых их корректность строго не доказана. 

\section{Генетические алгоритмы}

Генетические алгоритмы являются семейством в множестве эвристических алгоритмов. Впервые такой алгоритм был предложен А. Фразером. (см \cite{Фразер1970})
%эффективно применяются для решения одномерных рюкзаков - найти пруфы.
Алгоритм является итеративным.
Генетический алгоритм моделирует естественные процессы эволюции популяции, а именно - мутацию и скрещивание.
Решение задачи с помощью такого алгоритма требует нескольких предварительных этапов:
\begin{itemize}
	\item Выбор кодирования генотипа.\\
На этом этапе нужно выбрать способ кодирования генотипа, который будет эффективен для данной задачи. Такой генотип должен однозначно моделировать сущность, рассматриваемую в задаче.
	\item Выбор начального приближения.\\
Для запуска итерационного процесса требуется создать начальное множество - пул генотипов.
	\item Выбор мутации.\\
На каждой итерации алгоритма часть пула генотипов подвергется мутациям, то есть определенным образом изменяются их составляющие.
	\item Выбор механизма скрещивания (кроссинговера).\\
После мутации происходит создание новых генотипов из частей старых с сохранением признаков родителя. 
	\item Выбор функции оценки(фитнесс-функции).\\
Такая функция позволяет оценивать генотипы с точки зрения их близости к оптимальному решению и отбирать из них лучшие на каждой итерации.
\end{itemize}
\section{Выбор этапов}%ПЕРЕДЕЛАЙ И ПРОВЕРЬ
Наиболее естественным кодированием отдельного решения задачи о рюкзаке в генотип является бинарная последовательность длины N, состоящая из нулей и единиц.
 Каждый i-й элемент такой последовательности является индикатором вхождения i-го предмета в текущее решение. Такая модель требует наличия проверки условия \ref{Valid}
 \\ Для генерации начального приближения можно воспользоваться жадным алгоритмом, который сначала заполня
 
\section{Особенности для даннной задачи}%Тоже переделать.