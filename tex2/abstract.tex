\pagestyle{plain}
\chapter*{РЕФЕРАТ}
\thispagestyle{empty}   % отменить вывод номера страницы
Дипломная работа: \pageref*{LastPage}~с., \totfig~рис., \tottab~табл., \totref~источников.
Ключевые слова: ГЕНЕТИЧЕСКИЕ АЛГОРИТМЫ, МАШИННОЕ ОБУЧЕНИЕ, NP-ПОЛНЫЕ ЗАДАЧИ, ЗАДАЧА О РЮКЗАКЕ, ЛИНЕЙНОЕ ПРОГРАММИРОВАНИЕ, МНОГОМЕРНЫЙ РЮКЗАК
Объект исследования - задача о многомерном рюкзаке
Цель работы - исследование применения генетического алгоритма к задаче о многомерном рюкзаке.
Результатом работы является программа, реализующая генетический алгоритм на языке C\# и его модификации.
Разработаная программа позволяет найти точные максимумы для малых наборов предметов(<50 предметов) в 100\% случаев и для больших(100 предметов) - в 43,3\% случаев. В остальных случаях программа находит локальный максимум, различающий с исходным не более чем на 0,21\%.

\cleardoublepage               %  номер страницы в оглавлении не выводится
\pagenumbering{gobble}         %  номер страницы в оглавлении не выводится